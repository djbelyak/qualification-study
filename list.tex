\documentclass[a4paper,12pt]{extreport}

    \usepackage{cmap}                               % Улучшенный поиск русских слов в полученном pdf-файле
\defaulthyphenchar=127                          % Если стоит до fontenc, то переносы не впишутся в выделяемый текст при копировании его в буфер обмена 
\usepackage[T1,T2A]{fontenc} % Поддержка русских букв
\usepackage[utf8]{inputenc}[2014/04/30]         % Кодировка utf8
\usepackage[english, russian]{babel}[2014/03/24]% Языки: русский, английский

\usepackage{indentfirst} % Красная строка		% Подключаемые пакеты
    \input{styles}			% Пользовательские стили
    
    \begin{document}
    \centering

    \section*{Список опубликованных научных работ}

    {Белявцев Иван Павлович}

    {\it Статьи, опубликованные в перечне ведущих рецензируемых научных журналов, рекомендованных ВАК, и приравненных к ним изданиям:}

\begin{enumerate}
    \item Approximation of the criticality margin of WWR-c reactor using artificial neuron networks. / I. Belyavtsev, D. Legchikov, S. Starkov [и др.] // Journal of Physics: Conference Series. 2018. Т. 945, № 1. С.~012–031. (Scopus)
    \item Белявцев И.П., Старков С.О. Программный комплекс оценки запаса реактивности реактора ВВР-ц. // Известия высших учебных заведений. Ядерная энергетика. 2018. № 2. С.~58–66.
\end{enumerate}

{\it Статьи в сборниках научных трудов и сборниках трудов конференции:}

\begin{enumerate}[resume]
    \item Белявцев И.П., Старков С.О. Моделирование и оптимизация эксплутационных процессов на атомных электростанциях с использованием методов искусственного интеллекта. // Наукоемкие технологии в приборо- и машиностроении и развитие инновационной деятельности в ВУЗе: материалы Всероссийской научно-технической конференции, 25-27 ноября 2014 г. Т. 4. М.: Издательство МГТУ им. Н.Э. Баумана, 2014. С.~4–9.
    \item Построение нейросетевой модели реактора ВВР-ц для прогнозирования запаса критичности. / И.П. Белявцев, С.О. Старков, Д.К. Легчиков [и др.] // Наукоемкие технологии в приборо- и машиностроении и развитие инновационной деятельности в ВУЗе: материалы Всероссийской научно-технической конференции, 25-27 ноября 2014 г. Т. 4. М.: Издательство МГТУ им. Н.Э. Баумана, 2014. С.~10–15.
    \item Белявцев И.П., Старков С.О. Моделирование эксплутационных процессов ядерных реакторов с использованием методов искусственного интеллекта. // Научная сессия НИЯУ МИФИ-2015. Аннотации докладов. Т. 2. М.: НИЯУ МИФИ, 2015. С.~271.
    \item Прогнозирование запаса критичности реактора ВВР-ц методом нейросетевого моделирования. / И.П. Белявцев, С.О. Старков, Д.К. Легчиков [и др.] // Научная сессия НИЯУ МИФИ-2015. Аннотации докладов. Т. 2. М.: НИЯУ МИФИ, 2015. С.~272.
    \item Белявцев И.П., Старков С.О., Колесов В.В. Прогнозирование запаса критичности реактора ВВР-ц методом нейросетевого моделирования. // XIV Международная конференция «Безопасность АЭС и подготовка кадров». Тезисы докладов. Обнинск: ИАТЭ НИЯУ МИФИ, 2015. С.~138–139.
    \item Аппроксимация запаса критичности реактора ВВР-ц с использованием исскусственной нейронной сети. / И.П. Белявцев, Д.К. Легчиков, С.О. Старков [и др.] // Современные проблемы физики и технологий. VI-я Международная молодежная научная школа-конференция, 17-21 апреля 2017 г.: Тезисы докладов. Часть 1. М.: НИЯУ МИФИ, 2017. С.~80–81.
    \item Белявцев И.П., Старков С.О. Поиск оптимальной последовательности событий автоматной модели с использованием искусственных нейронных сетей. // Современные проблемы физики и технологий. VII-я Международная молодежная научная школа-конференция, 16-21 апреля 2017 г.: Тезисы докладов. Часть 2.  М.: НИЯУ МИФИ, 2018. С.~346–347.
\end{enumerate}
\end{document}
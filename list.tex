\documentclass[a4paper,12pt]{extreport}

    %%% Поля и разметка страницы %%%
\usepackage{lscape}		% Для включения альбомных страниц
\usepackage{geometry}	% Для последующего задания полей

%%% Кодировки и шрифты %%%
\usepackage{cmap}						% Улучшенный поиск русских слов в полученном pdf-файле
\usepackage[T2A]{fontenc}				% Поддержка русских букв
\usepackage[utf8]{inputenc}				% Кодировка utf8
\usepackage[english, russian]{babel}	% Языки: русский, английский
%\usepackage{pscyr}						% Красивые русские шрифты
%\usepackage{cmcyr}

%%% Математические пакеты %%%
\usepackage{amsthm,amsfonts,amsmath,amssymb,amscd} % Математические дополнения от AMS

%%% Оформление абзацев %%%
\usepackage{indentfirst} % Красная строка

%%% Цвета %%%
\usepackage[usenames]{color}
\usepackage{color}
%\usepackage{colortbl}

%%% Таблицы %%%
\usepackage{longtable}					% Длинные таблицы
\usepackage{multirow,array}	% Улучшенное форматирование таблиц

%%% Общее форматирование
\usepackage[singlelinecheck=off,center]{caption}	% Многострочные подписи
\usepackage{soul}									% Поддержка переносоустойчивых подчёркиваний и зачёркиваний

%%% Библиография %%%
\usepackage{cite} % Красивые ссылки на литературу

%%% Гиперссылки %%%
\usepackage[linktocpage=true,plainpages=false,pdfpagelabels=false]{hyperref}

%%% Изображения %%%
\usepackage{graphicx} % Подключаем пакет работы с графикой

%%% Оглавление %%%
\usepackage{tocloft}

%%% Исходные коды %%%
\usepackage{listings} 
\definecolor{mygreen}{rgb}{0,0.6,0}
\definecolor{mygray}{rgb}{0.5,0.5,0.5}
\definecolor{mymauve}{rgb}{0.58,0,0.82}
\lstset{ %
  backgroundcolor=\color{white},   % choose the background color; you must add \usepackage{color} or \usepackage{xcolor}
  basicstyle=\footnotesize,        % the size of the fonts that are used for the code
  breakatwhitespace=false,         % sets if automatic breaks should only happen at whitespace
  breaklines=true,                 % sets automatic line breaking
  captionpos=b,                    % sets the caption-position to bottom
  commentstyle=\color{mygreen},    % comment style
  deletekeywords={...},            % if you want to delete keywords from the given language
  escapeinside={\%*}{*)},          % if you want to add LaTeX within your code
  extendedchars=true,              % lets you use non-ASCII characters; for 8-bits encodings only, does not work with UTF-8
  keepspaces=true,                 % keeps spaces in text, useful for keeping indentation of code (possibly needs columns=flexible)
  keywordstyle=\color{blue},       % keyword style
  language=Octave,                 % the language of the code
  morekeywords={*,...},            % if you want to add more keywords to the set
  numbers=left,                    % where to put the line-numbers; possible values are (none, left, right)
  numbersep=5pt,                   % how far the line-numbers are from the code
  numberstyle=\tiny\color{mygray}, % the style that is used for the line-numbers
  rulecolor=\color{black},         % if not set, the frame-color may be changed on line-breaks within not-black text (e.g. comments (green here))
  showspaces=false,                % show spaces everywhere adding particular underscores; it overrides 'showstringspaces'
  showstringspaces=false,          % underline spaces within strings only
  showtabs=false,                  % show tabs within strings adding particular underscores
  stepnumber=1,                    % the step between two line-numbers. If it's 1, each line will be numbered
  stringstyle=\color{mymauve},     % string literal style
  tabsize=2,                       % sets default tabsize to 2 spaces
  title=\lstname                   % show the filename of files included with \lstinputlisting; also try caption instead of title
}		% Подключаемые пакеты
    %%% Макет страницы %%%
\geometry{a4paper,top=2cm,bottom=2cm,left=25mm,right=1cm}

%%% Кодировки и шрифты %%%
%\renewcommand{\rmdefault}{ftm} % Включаем Times New Roman


%%% Выравнивание и переносы %%%
\sloppy					% Избавляемся от переполнений
\clubpenalty=10000		% Запрещаем разрыв страницы после первой строки абзаца
\widowpenalty=10000		% Запрещаем разрыв страницы после последней строки абзаца

%%% Библиография %%%
\makeatletter
\bibliographystyle{utf8gost705u}	% Оформляем библиографию в соответствии с ГОСТ 7.0.5
\renewcommand{\@biblabel}[1]{#1.}	% Заменяем библиографию с квадратных скобок на точку:
\makeatother

%%% Изображения %%%
\graphicspath{{images/}} % Пути к изображениям

%%% Цвета гиперссылок %%%
\definecolor{linkcolor}{rgb}{0.9,0,0}
\definecolor{citecolor}{rgb}{0,0.6,0}
\definecolor{urlcolor}{rgb}{0,0,1}
\hypersetup{
    colorlinks, linkcolor={linkcolor},
    citecolor={citecolor}, urlcolor={urlcolor}
}

%%% Оглавление %%%
\renewcommand{\cftchapdotsep}{\cftdotsep}

%%% Определения %%%
\newtheorem{Def}{Определение}[chapter]

% 1,5 интервал
\linespread{1.3}
			% Пользовательские стили
    
    \begin{document}
    \centering

    \section*{Список опубликованных научных работ}

    {Белявцев Иван Павлович}

    {\it Статьи, опубликованные в перечне ведущих рецензируемых научных журналов, рекомендованных ВАК, и приравненных к ним изданиям:}

\begin{enumerate}
    \item Approximation of the criticality margin of WWR-c reactor using artificial neuron networks. / I. Belyavtsev, D. Legchikov, S. Starkov [и др.] // Journal of Physics: Conference Series. 2018. Т. 945, № 1. С.~012–031. (Scopus)
    \item Белявцев И.П., Старков С.О. Программный комплекс оценки запаса реактивности реактора ВВР-ц. // Известия высших учебных заведений. Ядерная энергетика. 2018. № 2. С.~58–66.
\end{enumerate}

{\it Статьи в сборниках научных трудов и сборниках трудов конференции:}

\begin{enumerate}[resume]
    \item Белявцев И.П., Старков С.О. Моделирование и оптимизация эксплутационных процессов на атомных электростанциях с использованием методов искусственного интеллекта. // Наукоемкие технологии в приборо- и машиностроении и развитие инновационной деятельности в ВУЗе: материалы Всероссийской научно-технической конференции, 25-27 ноября 2014 г. Т. 4. М.: Издательство МГТУ им. Н.Э. Баумана, 2014. С.~4–9.
    \item Построение нейросетевой модели реактора ВВР-ц для прогнозирования запаса критичности. / И.П. Белявцев, С.О. Старков, Д.К. Легчиков [и др.] // Наукоемкие технологии в приборо- и машиностроении и развитие инновационной деятельности в ВУЗе: материалы Всероссийской научно-технической конференции, 25-27 ноября 2014 г. Т. 4. М.: Издательство МГТУ им. Н.Э. Баумана, 2014. С.~10–15.
    \item Белявцев И.П., Старков С.О. Моделирование эксплутационных процессов ядерных реакторов с использованием методов искусственного интеллекта. // Научная сессия НИЯУ МИФИ-2015. Аннотации докладов. Т. 2. М.: НИЯУ МИФИ, 2015. С.~271.
    \item Прогнозирование запаса критичности реактора ВВР-ц методом нейросетевого моделирования. / И.П. Белявцев, С.О. Старков, Д.К. Легчиков [и др.] // Научная сессия НИЯУ МИФИ-2015. Аннотации докладов. Т. 2. М.: НИЯУ МИФИ, 2015. С.~272.
    \item Белявцев И.П., Старков С.О., Колесов В.В. Прогнозирование запаса критичности реактора ВВР-ц методом нейросетевого моделирования. // XIV Международная конференция «Безопасность АЭС и подготовка кадров». Тезисы докладов. Обнинск: ИАТЭ НИЯУ МИФИ, 2015. С.~138–139.
    \item Аппроксимация запаса критичности реактора ВВР-ц с использованием исскусственной нейронной сети. / И.П. Белявцев, Д.К. Легчиков, С.О. Старков [и др.] // Современные проблемы физики и технологий. VI-я Международная молодежная научная школа-конференция, 17-21 апреля 2017 г.: Тезисы докладов. Часть 1. М.: НИЯУ МИФИ, 2017. С.~80–81.
    \item Белявцев И.П., Старков С.О. Поиск оптимальной последовательности событий автоматной модели с использованием искусственных нейронных сетей. // Современные проблемы физики и технологий. VII-я Международная молодежная научная школа-конференция, 16-21 апреля 2017 г.: Тезисы докладов. Часть 2.  М.: НИЯУ МИФИ, 2018. С.~346–347.
\end{enumerate}
\end{document}
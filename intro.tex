\chapter*{Введение}							% Заголовок
\addcontentsline{toc}{chapter}{Введение}	% Добавляем его в оглавление

\textbf{Актуальность темы исследования.}

В настоящее время в мире эксплуатируется 451 ядерная энергетическая устновка \cite{iaea}.
Согласно прогнозам экспертов, к 2040 году ожидается  60\%-ный рост мощности суммарной мощности ядерных электростанций.
Таким образом в 20-летней перспективе суммарные мировые запасы отработанного ядерного топлива составят порядка 700 тыс. тонн \cite{tends-2015}.
Большое количество отработанного ядерного топлива представляет собой не только серьезную проблему по организации дорогостоющих процессов переработки и захоронения, но и также ставит задачи по обеспечению экологической и антитеррористической защите.

Существующий открытый ядреный топливный цикл обладает низкой эффективностью использования урана -- менее 1\%.
В настоящее время РосАтом активно развивает технологии закрытого ядерного топливного цикла, которые позволят переиспользовать более 95\% ядерного топлива, что поможет значительно снзить объем захоронений ядерных отходов.
\cite{cycle}

Замкнтый ядерный цикл требует постройки большого количества реакторов на быстрых нейтронах типа СВБР-100, БН-1200, БРЕСТ. Кроме того требуется развернуть в больших масштабах производство топлива для реакторов на быстрых нейтронах. Эти задачи запланированы РосАтомом как среднесрочные и долгосрочные. 
А в краткосрочной перспективе, существует задача оптимизации технологий существующих ядерных энергетических установок.\cite{cycle}

Одним из направлений оптимизации существующих ядерных энергетических установок является достижение большей глубины выгорания топлива.
При этом наименее затратным способом является оптимизация порядка перегруок ядерного топлива.

Существующие методы и программные средства моедлилрования и расчета реакторов пологаются на устоявшиеся последовательности перегрузки ядерного топлива или требуют ручного использования.
Решаемая научная и практическая задача состоит в разработке и реализации метода оптимизации технологических процессов ЯЭУ, а также программных средств, реализующих данный метод.



\textbf{Объект исследования} --- эсплутационные процесс перегрузки ядерного топлива в исследовательских и энергетических ядерных реакторах.

\textbf{Предмет исследования} --- интеллектуальные методы и программные средства оптимизации эксплутационных процессов ядерных реакторов.

\textbf{Целью исследования} является решение задач автоматической оптимизации порядка перезагрузки ядерного топлива ядерных реакторов.

Для достижения поставенной цели тербуется решение следующих \textbf{задач}:

\begin{enumerate}
    \item Разработать описательную модель эксплутационных процессов ядерных реакторов;
    \item Исследовать и разработать алгоритмы поиска оптимальной последовательности экплутационного процесса;
    \item Разработать алгоритм аппроксимации эмерджентных свойств для описательной модели.
\end{enumerate}

\textbf{Методы исследования.} 
Работа базируется на методах теории автоматов, методах теории графов, численных методах и искуственных нейронных сетях.

\textbf{Научная новизна} результатов исследования состоит в следующем
\begin{enumerate}
    \item Впервые предложен подход к моделированию эксплутационных процессов ядерных реакторов на основе иерархических автоматов;
    \item Представлены алгоритмы поиска оптимальной последовательности экспутационного процесса;
    \item Предложен и реализован метод построения аппроксимации эмерджентных свойств c использованием искуственных нейронных сетей;
    \item Реализован программный комплекс аппроксимации запаса реактивности реактора ВВР-ц.
\end{enumerate}

\textbf{Практическая значимость. }

Научные и практические результаты диссертации использованы при разработке программного комплекса аппроксимации запаса реактивности реактора ВВР-ц.
Предполагаемым потребителем результатов диссертации является филиал АО <<НИФХИ им. Л.Я.Карпова>> в г. Обнинск, эксплуатирующий реактор ВВР-ц.

\textbf{Основные положения, выносимые на защиту.}

\begin{enumerate}
    \item Подход к построению автоматной модели эксплутационных процессов ядерных реакторов;
    \item Алгоритмы поиска оптимальной последовательности процесса в автоматной модели;
    \item Метод аппроксимации эмерджентных свойств для автоматной модели на основе искусственных нейронных сетей;
    \item Программный комплекс аппроксимации запаса реактивности реактора ВВР-ц.
\end{enumerate}

\textbf{Апробация работы.}

Материалы диссертации докладывались на следующих всероссийских и международных конференциях:

\begin{enumerate}
    \item Всероссийская научно-техническая конференция <<Наукоемкие технологии в приборо- и машиностроении и развитие инновационной деятельности в ВУЗе>>, г.~Калуга, 25-27 ноября 2014 г.;
    \item Научная сессия НИЯУ МИФИ, г.~Москва, 16-20 февраля 2015 г.;
    \item XIV Международная конференция <<Безопасность АЭС и подготовка кадров>>, г.~Обнинск, 25-27 ноября 2015 г.;
    \item VI-я Международная молодежная научная школа-конференция <<Современные проблемы физики и технологий>>, г.~Москва, 17-21 апреля 2017 г.;
    \item VII-я Международная молодежная научная школа-конференция <<Современные проблемы физики и технологий>>, г.~Москва, 16-21 апреля 2018 г.
\end{enumerate}

\textbf{Публикации.}

Основные результаты исследованиий изложены в 9 научных трудах, 2 из которых опубликованы в изданиях, рекомендованных ВАК, или приравненных к ним изданиям.

% \textbf{Структура и объем работы.}

\section*{Содержание работы}

Во \textbf{введении} обоснована актаульность темы научной работы, сформулированы цели и задачи, перечислены полученные в работе новые результаты, их практическая значимость, представлены положения, выносимые на защиту, описана структура диссертации.

В \textbf{первой главе} приведен обзор устройства ядерного реактора на примере реактора БН-350.
Рассмотрны основные конструкционные элементы, а также произведен разбор процесса перегрузки ядерного топлива.

Выполнен сравнительный анализ существущих программных средств, применяемых для планирования перегрузки: TRIGEX, САПФИР\_95, RC\_ВВЭР, REPRORYV.
Выявлено, что существующие программные комплексы способны проводить расчет либо только для стационарных схем перегрузки, либо по заданной вручную схеме.

Сделан вывод о необходимости алгоритмического решения задачи оптимизации технологических процессов ядерных реакторов на примере процесса перегузки ядерного топлива.

Во \textbf{второй главе} сформулирована постановка задачи оптимизации технологического процесса.
Предложен общий подход к решению задачи оптимизации технологического процесса.
Определены следующие входные данные: описание конструкции реактора, данные для физического моделирования и функция оптимизации.
В качестве выходных данных определн оптимальный порядок технологического процесса.

Рассмотрены основные положения теории автоматов.
Введено понятие автоматной модели (ТА-модели).
Рассмотрен подход к построению композиции ТА-моделей.

Показан пример построения ТА-модели фиктивного реактора.

В \textbf{третьей главе} рассмотрен подход к поиску оптимальной последовательности в автоматной модели.
Показана возможность представления детерминированного конечного автомата в виде направленного графа (диаграммы переходов).
Таким образом, задача поиска оптимальной последовательности автоматной модели сведена к задачи посика кратчайшего пути в графе.

Рассмотрены классические алгоритмы поиска пути в графе: алгоритм Дейкстры, алгоритм A* и волновой алгоритм.
Для каждого из алгоритмов приведен пример.
Из класических алгоритмов наиболее подходящим является алгоритм А*, применение которого рекомендуется для небольших и средних ТА-моделей.

Для больших моделей предложен нейросетевой алгоритм для поиска последовательности.

В \textbf{четвертой главе} рассмотрена возможность аппроксимации запаса реактивности реактора с помощью полносвязной искусственной нейронной сети; обучены две искусственные нейронные сети на разных наборах данных (на модельных, полученных с помощью прецизионной модели реактора, и на измеренных данных реальных кампаний). 

Показано, что обе аппроксимации обладают достаточной точностью для проведения предварительных расчетов запаса реактивности.
По итогам вычислительных экспериментов максимальная относительная ошибка аппроксимации составила 3,13 и 3,56\% соответственно.

На основе обученных искусственных нейронных сетей создан программный комплекс оценки запаса реактивности реактора ВВР-ц.
Комплекс позволяет в удобной и наглядной форме получить предсказанное нейронной сетью значение запаса реактивности, а также пополнить обучающую выборку новыми данными для обучения.

Программный комплекс для оценки запаса реактивности готов для тестирования персоналом реактора ВВР-ц.
Параллельно с тестированием в данный программный комплекс можно внести ряд изменений, повышающих удобство и безопасность эксплуатации: шифрование данных в обучающей выборке; авторизацию, аутентификацию и аккаунтинг пользователей; возможность ручного редактирования обучающей выборки.

Допустимо использование данного программного комплекса в виде компонента системы автоматического планирования перезагрузки реактора. 
С незначительными изменениями комплекс можно применять для реакторных установок других типов.

В \textbf{Приложении} представленны исходные тексты программных продуктов, разработанных в ходе научной работы.


\section*{Публикации по теме работы}

{\it Статьи, опубликованные в перечне ведущих рецензируемых научных журналов, рекомендованных ВАК, и приравненных к ним изданиям:}

\begin{enumerate}
    \item Approximation of the criticality margin of WWR-c reactor using artificial neuron networks. / I. Belyavtsev, D. Legchikov, S. Starkov [и др.] // Journal of Physics: Conference Series. 2018. Т. 945, № 1. С.~012–031. (Scopus)\cite{iop-2018}
    \item Белявцев И.П., Старков С.О. Программный комплекс оценки запаса реактивности реактора ВВР-ц. // Известия высших учебных заведений. Ядерная энергетика. 2018. № 2. С.~58–66. \cite{npe-2018}
\end{enumerate}

{\it Статьи в сборниках научных трудов и сборниках трудов конференции:}

\begin{enumerate}[resume]
    \item Белявцев И.П., Старков С.О. Моделирование и оптимизация эксплутационных процессов на атомных электростанциях с использованием методов искусственного интеллекта. // Наукоемкие технологии в приборо- и машиностроении и развитие инновационной деятельности в ВУЗе: материалы Всероссийской научно-технической конференции, 25-27 ноября 2014 г. Т. 4. М.: Издательство МГТУ им. Н.Э. Баумана, 2014. С.~4–9. \cite{modeling-2014}
    \item Построение нейросетевой модели реактора ВВР-ц для прогнозирования запаса критичности. / И.П. Белявцев, С.О. Старков, Д.К. Легчиков [и др.] // Наукоемкие технологии в приборо- и машиностроении и развитие инновационной деятельности в ВУЗе: материалы Всероссийской научно-технической конференции, 25-27 ноября 2014 г. Т. 4. М.: Издательство МГТУ им. Н.Э. Баумана, 2014. С.~10–15. \cite{criticality-2014}
    \item Белявцев И.П., Старков С.О. Моделирование эксплутационных процессов ядерных реакторов с использованием методов искусственного интеллекта. // Научная сессия НИЯУ МИФИ-2015. Аннотации докладов. Т. 2. М.: НИЯУ МИФИ, 2015. С.~271.\cite{modeling-2015}
    \item Прогнозирование запаса критичности реактора ВВР-ц методом нейросетевого моделирования. / И.П. Белявцев, С.О. Старков, Д.К. Легчиков [и др.] // Научная сессия НИЯУ МИФИ-2015. Аннотации докладов. Т. 2. М.: НИЯУ МИФИ, 2015. С.~272.\cite{criticality-2015}
    \item Белявцев И.П., Старков С.О., Колесов В.В. Прогнозирование запаса критичности реактора ВВР-ц методом нейросетевого моделирования. // XIV Международная конференция «Безопасность АЭС и подготовка кадров». Тезисы докладов. Обнинск: ИАТЭ НИЯУ МИФИ, 2015. С.~138–139. \cite{criticality-2015-2}
    \item Аппроксимация запаса критичности реактора ВВР-ц с использованием исскусственной нейронной сети. / И.П. Белявцев, Д.К. Легчиков, С.О. Старков [и др.] // Современные проблемы физики и технологий. VI-я Международная молодежная научная школа-конференция, 17-21 апреля 2017 г.: Тезисы докладов. Часть 1. М.: НИЯУ МИФИ, 2017. С.~80–81. \cite{mephi-2017}
    \item Белявцев И.П., Старков С.О. Поиск оптимальной последовательности событий автоматной модели с использованием искусственных нейронных сетей. // Современные проблемы физики и технологий. VII-я Международная молодежная научная школа-конференция, 16-21 апреля 2017 г.: Тезисы докладов. Часть 2.  М.: НИЯУ МИФИ, 2018. С.~346–347. \cite{mephi-2018}
\end{enumerate}
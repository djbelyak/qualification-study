\chapter*{Введение}							% Заголовок
\addcontentsline{toc}{chapter}{Введение}	% Добавляем его в оглавление

На сегодняшний день ядерная энергия является одной из самых перспективных и самых опасных видов энергии.
Ядерная энергия, несмотря на огромный объем капитальных затрат на создание и утилизацию атомной электростанции (АЭС), является довольно дешевой энергией, так как из малого объема ядерного топлива получается огромное количество энергии.
Для выработки энергии $1$ МВт$\cdot$сут требуется всего лишь $1,2$ г делящегося изотопа (урана$-235$).
При сравнении энергетических эквивалентов органического и ядерного топлива обнаруживается, что несколько граммов делящегося изотопа урана$-235$ примерно равны $1$ т нефти (точнее $4$~г~$^{235}U\approx 1$ т нефти)!

Другая, не менее важная причина состоит в том, что имеющихся ресурсов органического топлива при нынешних тенденциях энергопотребления хватит, по различным прогнозам, на 50-200 лет, а ресурсов урана (как урана-235, так и урана-238 с учетом построения замкнутого топливного цикла) хватит примерно на 10 000 лет. Кроме запасов урана, на Земле имеются еще запасы тория, объем которых, по оценкам, сопоставим с запасами урана, а возможно, в несколько раз их превосходит.\cite{Ukr}

Несмотря на все эти неоспоримые преимущества атомная энергетика таит в себе ряд опаснейших проблем.
Примером, таких проблем может служить опасность теплового взрыва реактора, необходимость захоронения ядерных отходов, усталость капитальных конструкций реактора.
При этом если задачу предотвращения опасности теплового взрыва можно считать во многих аспектах решенной, то задача уменьшения усталости капитальных конструкций реактора является актуальной по сей день. На сегодняшний день ядерная энергия является одной из самых перспективных и самых опасных видов энергии.
Ядерная энергия, несмотря на огромный объем капитальных затрат на создание и утилизацию атомной электростанции (АЭС), является довольно дешевой энергией, так как из малого объема ядерного топлива получается огромное количество энергии.
Для выработки энергии $1$ МВт$\cdot$сут требуется всего лишь $1,2$ г делящегося изотопа (урана$-235$).
При сравнении энергетических эквивалентов органического и ядерного топлива обнаруживается, что несколько граммов делящегося изотопа урана$-235$ примерно равны $1$ т нефти (точнее $4$~г~$^{235}U\approx 1$ т нефти)!

Другая, не менее важная причина состоит в том, что имеющихся ресурсов органического топлива при нынешних тенденциях энергопотребления хватит, по различным прогнозам, на 50-200 лет, а ресурсов урана (как урана-235, так и урана-238 с учетом построения замкнутого топливного цикла) хватит примерно на 10 000 лет. Кроме запасов урана, на Земле имеются еще запасы тория, объем которых, по оценкам, сопоставим с запасами урана, а возможно, в несколько раз их превосходит.
\cite{Ukr}

Несмотря на все эти неоспоримые преимущества атомная энергетика таит в себе ряд опаснейших проблем.
Примером, таких проблем может служить опасность теплового взрыва реактора, необходимость захоронения ядерных отходов, усталость капитальных конструкций реактора.
При этом если задачу предотвращения опасности теплового взрыва можно считать во многих аспектах решенной, то задача уменьшения усталости капитальных конструкций реактора является актуальной по сей день. 

Наибольшие воздействия на капитальные конструкции происходят во время планово-про\-фи\-лак\-ти\-чес\-ких ремонтов (ППР) реактора. 
На сегодняшний день отсутствуют методологии по определению наиболее оптимального плана ППР с точки зрения воздействия на капитальные конструкции, а так же по времени проведения ППР.

\textbf{Целью} данной работы является разработка методологии моделирования и построения оптимального порядка планово-профилактического ремонта реакторов с использованием методов искусственного интеллекта.

Для~достижения поставленной цели необходимо было решить следующие задачи:
\begin{enumerate}
  \item Разработать формальную информационную модель, описывающую процесс перезагрузки реактора
  \item Исследовать метод построения оптимального порядка процесса перезагрузки реактора
  \item Разработать способ адаптивного моделирования физических параметров активной зоны реактора
  \item Разработать программный комплекс, реализующий моделирование и построение оптимального порядка ППР реакторов
\end{enumerate}

\textbf{Основные положения, выносимые на~защиту:}
\begin{enumerate}
  \item Разработана формальная информационная модель, описывающая процесс перезагрузки реактора, на основе аппарата теории автоматов
  \item Предложен метод оптимизации порядка ППР с использованием графа переходов и алгоритма А*
  \item Разработан способ адаптивного моделирования физических параметров активной зоны реактора с использованием искусственных нейронных сетей прямого распространения.
  \item Реализован программный комплекс, реализующий моделирование и построение оптимального порядка ППР реакторов
\end{enumerate}

\textbf{Научная новизна:}
\begin{enumerate}
  \item Было выполнено оригинальное исследование процесса ППР различных реакторов
  \item Впервые была предложена и опробована формальная информационная модель ППР на основе аппарата теории графов
  \item Впервые был предложен и опробован адаптивный способ моделирования физических параметров активной зоны реактора с использованием искусственных нейронных сетей
\end{enumerate}

\textbf{Научная и практическая значимость}.
Результаты настоящей работы могут быть применены для моделирования и построения оптимального порядка планово-профилактического ремонта реакторов. 

%\textbf{Степень достоверности} полученных результатов обеспечивается \ldots Результаты находятся в соответствии с результатами, полученными другими авторами.

\textbf{Апробация работы.}
Основные результаты работы докладывались~на конференции <<Применение кибернетических методов в решении проблем общества XXI века>> (Обнинск, 2014).

%\textbf{Личный вклад.} Автор принимал активное участие \ldots

\textbf{Публикации.} Основные результаты по теме диссертации изложены в тезисах докладов~\cite{conf-iate-2014}.

\textbf{Объем и структура работы.} Диссертация состоит из~введения, пяти глав, заключения и~двух приложений.
Полный объем диссертации составляет 74~страницы с~31~рисунком и~13~таблицами. Список литературы содержит 24~наименования.

\clearpage
\chapter*{Заключение}						% Заголовок
\addcontentsline{toc}{chapter}{Заключение}	% Добавляем его в оглавление

Основные результаты работы заключаются в следующем.
\begin{enumerate}
  \item На основе исследований конструкции и порядка планово-профилактического ремонта реактора была предложена формальная информационная модель реактора.
  Отличительной особенностью предложенной модели является представление узлов и агрегатов реактора в виде автоматов.
  Это позволяет описывать процесс перезагрузки с любой детализацией, используя идентичные информационные структуры.
  Таким образом данная методология может быть применена к любому типу реакторов. 
  \item Исследование методов поиска кратчайшего пути в графах позволили предложить метод построения оптимального порядка перезагрузки.
  Модель реактора на основе теории автоматов (ТА-модель) допускает представление в виде графа, что позволяет найти путь от начального состояния автомата до конечного методами теории графов.
  Было предложено использовать информированный поиск на основе алгоритма А*.
  Задавая соответствующую эвристическую функцию поиска можно производить оптимизацию по различным критериям, не изменяя модели и методологии поиска. 
  \item Для предварительного моделирования физических параметров активной зоны было предложено использовать искусственную нейронную сеть.
  Данный подход требует предварительного обучения на множестве значений, однако позволяет избежать длительных расчетов на этапе поиска пути.
  Обучающая выборка может состоять не только из данных, рассчитанных по математическим моделям реактора, но и реальные данные предыдущих кампаний.
  Что в свою очередь позволяет учесть погрешности математических моделей.
  \item Для выполнения поставленных задач был разработан программный комплекс.
  Разработанный комплекс позволяет создавать и хранить различные ТА-модели, находить оптимальный путь с заданной эвристикой, а также моделировать некоторые параметры состояний ТА-модели при наличии обучающей выборки.
  Данный программный комплекс в полной мере иллюстрирует действие положений настоящей диссертации.
  
\end{enumerate}

Таким образом, предложенная методология расчета порядка перезагрузки реактора доказала свою перспективность.
Однако, для использования в атомной промышленности потребуется ряд серьезных испытаний и доработок, которые позволят повысить точность и производительность методологии.

\clearpage
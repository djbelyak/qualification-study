\chapter*{Заключение}						% Заголовок
\addcontentsline{toc}{chapter}{Заключение}	% Добавляем его в оглавление

% Положения выносимые на защиту

В рамках данной научной работы были достигнуты следующие результаты:
\begin{enumerate}
    \item Рассмотрены конструкция ядерного реактора и процесс перегрузки ядерного топлива;
    \item Исследованы возможности программных средств, применяемых для расчета процессов перегрузки;
    \item Обоснована необходимость разработки метода и программных средств для оптимизации эксплутационных процессов;
    \item Предложена автоматная модель описания эксплутационных процессов;
    \item Показана применимость автоматной модели для описания процесса перегрузки ядерного топлива;
    \item Показана применимость методов поиска в графе для поиска оптимальной последовательности ТА-модели;
    \item Предложен нейросетевой подход к поиску оптимальной последовательности для больших автоманых моделей;
    \item Предложено использование искусственной нейронной сети для аппроксимации эмерджентных свойств;
    \item Проведены эксперименты по построению нейросетевой аппроксимации для запаса реактивности реактора ВВР-ц;
    \item Разработан программный комплекс аппроксимации запаса реактивности реактора ВВР-ц.
\end{enumerate}

На защиту выносятся следующие основные положения:
\begin{enumerate}
    \item Разработан подход к построению автоматной модели эксплутационных процессов ядерных реакторов;
    \item Предложены алгоритмы поиска оптимальной последовательности процесса в автоматной модели (в том числе и с использованием искусственных нейронных сетей);
    \item Разработан и проверен метод аппроксимации эмерджентных свойств для автоматной модели на основе искусственных нейронных сетей;
    \item Разработан программный комплекс аппроксимации запаса реактивности реактора ВВР-ц.
\end{enumerate}

\begin{center}
\section*{Реферат}
\end{center}

\vspace{2em}
74 стр., 31 рис., 24 ист.

\vspace{2em}
ЯДЕРНЫЙ РЕАКТОР, ПЕРЕГРУЗКА, ТА-МОДЕЛЬ, АВТОМАТ, ГРАФ, АЛГОРИТМЫ ПОИСКА, ИСКУССТВЕННАЯ НЕЙРОННАЯ СЕТЬ, ТОПОЛОГИИ, АЛГОРИТМЫ ОБУЧЕНИЯ, \textit{PYTHON}, \textit{REST}, ОБЛАЧНЫЕ ВЫЧИСЛЕНИЯ


\vspace{2em}
Объектом исследования являются процессы перегрузки топлива в ядерных энергетических реакторах.

Цель работы --- создание метода моделирования и оптимизации процесса перегрузки топлива в ядерных реакторах с использованием методов искусственного интеллекта.

В данной работе исследуется строение ядерного реактора и процесс перегрузки топлива.
Предлагается формальная информационная модель реактора на основе представления реактора как композиции автоматов (ТА-модель).
Исследуются алгоритмы поиска в графах и искусственные нейронные сети прямого распространения.
Проводится работа по созданию программного комплекса для оптимизации процесса перегрузки реактора, который использует представление ТА-модели, поиск по графу переходов ТА-моделей и искусственные нейронные сети для оценок физических параметров состояния ТА-модели.

Программный комплекс написан на языке программирования Python по архитектуре REST сервисов.
Возможна установка программного комплекса на облачные решения типа PaaS и IaaS.

\newpage